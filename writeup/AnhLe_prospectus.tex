\documentclass[12pt]{article}
% This first part of the file is called the PREAMBLE. It includes
% customizations and command definitions. The preamble is everything
% between \documentclass and \begin{document}.

\usepackage[margin=1in]{geometry}  % set the margins to 1in on all sides
\usepackage{graphicx}              % to include figures
\usepackage{amsmath}               % great math stuff
\usepackage{amsfonts}              % for blackboard bold, etc
\usepackage{amsthm}                % better theorem environments

\usepackage{rotating} % for sideway table
\usepackage{xcolor}
\usepackage{hyperref}
\hypersetup{
    colorlinks,
    linkcolor={red!50!black},
    citecolor={blue!50!black},
    urlcolor={blue!80!black}
}
\usepackage{cleveref}

\usepackage{array,tabularx}

\newenvironment{conditions*}
  {\par\vspace{\abovedisplayskip}\noindent
   \tabularx{\columnwidth}{>{$}l<{$} @{${}={}$} >{\raggedright\arraybackslash}X}}
  {\endtabularx\par\vspace{\belowdisplayskip}}
  
\usepackage{float}
\restylefloat{table}

% bibliography
\usepackage{natbib}
\bibpunct{(}{)}{;}{a}{}{,} % no comma between author and year

\title{Paper title}
\author{Anh Le}


\begin{document}
\maketitle

\section{Firm size and FDI}
\label{sec:background}
This is about FDI demand against domestic firms (as contrast with motivationally credible commitment)

Countries want FDI that is export oriented instead of import oriented
- The causal mechanism flows through the lobbying effort of firms
- Labor likes FDI either way

She argues that FDI entry regulation -> allow local firms to seize asset of foreign firms -> reduce FDI
- Theory: So this is preferred by firms. But why by countries, especially authoritarian regimes? There must be an interaction with the lobbying capacity of firms.
	+ This is also about FDI entry regulation as an independent variable
- Empirics: entry regulation is at the industry level. The outcome (FDI inflow) is at the country level

\end{document}
