This is about FDI demand against domestic firms (as contrast with motivationally credible commitment)

Countries want FDI that is export oriented instead of import oriented
- The causal mechanism flows through the lobbying effort of firms
- Labor likes FDI either way

She argues that FDI entry regulation -> allow local firms to seize asset of foreign firms -> reduce FDI
- Theory: So this is preferred by firms. But why by countries, especially authoritarian regimes? There must be an interaction with the lobbying capacity of firms.
	+ This is also about FDI entry regulation as an independent variable
- Empirics: entry regulation is at the industry level. The outcome (FDI inflow) is at the country level