\subsubsection{Discrimination in treatment of foreign and private firms}

In sum, the spillover effect of FDI depends on a strong domestic sector. Therefore, if a government is truly interested in FDI for its spillover and growth-enhancing effect, the government must be equally interested, if not more, in nurturing the absorptive capacity of domestic firms. Holding constant firm characteristics (size, sector, technological capacity, etc.), if we find that domestic firms receive worse treatment by the government, this can be evidence that the government is not primarily interested in the spillover effect of FDI, but potentially rent seeking.

Anticipating alternative explanations, there are reasons for countries to attract FDI other than growth and corruption, such as jobs, capital, and balance of payments. Fortunately, these alternative explanations can be controlled for. Moreover, these factors may account for the enthusiasm of the government towards FDI, but cannot fully explain the discrimination against domestic firms in favor of FDI. Indeed, many factors that are attractive to foreign firms (e.g. skilled labor force, good infrastructure, good governance) has high fixed cost but low marginal cost, and thus should be easily extended to domestic firms. Therefore, if corruption is not the government wants FDI, the domestic sector would benefit instead of being discriminated against.
