For this argument to be convincing, one must take into account alternative explanations, i.e. reasons besides corruption and growth that governments may want FDI for:
\begin{itemize} 
\item Employment: under strong pressure for employment generation, the government may want FDI purely for the jobs it brings instead of long-term economic growth. To account for this alternative explanation, I will control for the growth rate of the labor force. Since labor force growth is largely determined 18-20 years prior, it is plausibly exogenous to other variables in the current period and thus well-behaved statistically.  
\item Capital: In the early-stage of development, a country may deliberately pursue a capital-driven instead of technology-driven growth. To account for a country's immediate need for capital, I will control for the aggregate capital stock.
\item Election cycle: Much research has shown that the election calendar puts populist pressure on the incumbent government, leading to manipulation of macroeconomic factors such as the exchange rate \citep{Blomberg2001}. Similarly, one may argue that the government attracts FDI to generate positive headlines near election dates even though these FDI projects do not have a large impact on long-term growth. To account for this alternative explanation, I will control for whether a foreign firm establishes in an election year.
\end{itemize}
