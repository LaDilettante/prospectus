\subsection{Two-sided logit (TSL) model}
\label{sec:rd_tsl}

As discussed in \Cref{sec:theory_actors_and_choices}, here I will extend my model to consider a strategic firm using the TSL model. Designed to study the job market, the TSL is ``a two-sided approach [that] explicitly [combines] models of employers' and workers' preferences \dots with data on the characteristics or resources that each side values in the other'' \citep[117]{Logan1996}. In this section, I show how to frame the FDI investment location in the TSL framework, where the firm chooses the official based on his endowment and the official chooses the firm based on its ability to deliver spillover and private benefits. Extending my model in \Cref{sec:theory_actors_and_choices}, the TSL allows the firm to be profit-maximizing instead of profit-satisficing. Furthermore, both sides are also strategic, taking into account not only their own preferences but also offers from the other sides \citep{Logan1996a}.

The data is a random sample of firms and the countries that they invest in, including relevant firm-level and country-level covariates. The research design is to categorize those countries' officials according to the length of their time horizon. Then, for each category, I use the TSL to estimate the preference of the official for spillover versus private benefits. The hypothesis is that officials with a longer time horizon prefer spillover.

In the following pages, \Cref{sec:tsl_actor_utility} specifies the actors' utility functions. \Cref{sec:tsl_estimate} discusses how the TSL model is derived from these utility functions and how it can be estimated. Finally, \Cref{sec:tsl_hypothesis} elaborates on the hypotheses regarding the effect of time horizon and how to operationalize them for testing.

\subsubsection{The actors' utility functions}
\label{sec:tsl_actor_utility}


Using the notation from \citet{Logan1998}, we consider the utility function of the two actors, the official and the firm. Since there is only one official for a country in the model, in this section I will refer to country $j$ and official $j$ interchangeably. For the official $j$, the utility of having firm $i$ investing in his country is:

\begin{align}
U_j(i) &= \beta_j x_i + m_j + \epsilon_{1ij} \\
\end{align}

while the utility of not having firm $i$ investing is:

\begin{align}
U_j(\neg i) &= b_j + \epsilon_{0ij}
\end{align}

where

\begin{align*}
\beta_j &= \text{a vector of official $j$'s preference for relevant characteristics of firms} \\
x_i &= \text{a vector of firm $i$'s measured values on those characteristics} \\
b_j &= \text{the baseline utility of official $j$ without any firm investing} \\
\epsilon_{1ij}, \epsilon_{0ij} &= \text{unobserved components that influence official $j$'s utility}
\end{align*}

The official $j$ will evaluate each firm and then make an offer to firm $i$ to invest if $U_j(i) > U_j(\neg i)$. In our model, the relevant firm characteristics (i.e. $x_i$) that the official will consider are: the potential for spillover, private benefits, jobs, and capital. The corresponding $\beta$'s represent the official's preference for these characteristics. We are mainly interested in the $\beta$'s for spillover and private benefits.

On the other side, for firm $i$, the utility of investing in country $j$ is:

\begin{align}
V_i(j) &= \alpha_i w_{ij} + v_{ij}
\end{align}

where

\begin{align*}
\alpha_i &= \text{a vector of firm $i$'s preference for relevant characteristics of countries} \\
w_{ij} &= \text{a vector of country $j$ measured values on those characteristics} \\
v_{ij} &= \text{unobserved component that influences firm $i$'s utility}
\end{align*}

Firm $i$ evaluates all the countries that make an offer and chooses the one that brings the highest utility. In our model, the relevant country characteristics are: labor quality, infrastructure, and market size.

\subsubsection{Estimate the actors' preference}
\label{sec:tsl_estimate}

Our data contains a random sample of firms and the countries in which they invest. We want to find the parameters that maximize the likelihood of this observed data. This likelihood is:

\[
L = \prod_{i,j: \text{i is matched with j}} Pr(A_{ij})
\]

where $Pr(A_{ij})$ is the probability of a specific match between firm $i$ and country $j$. $Pr(A_{ij})$ can be calculated as follows:

\begin{align}
&Pr(A_{ij}) \\
&= \sum_{k=1}^R Pr(A_{ij}|S_{ik}) Pr(S_{ik}) \\
&= \sum_{k=1}^R Pr(A_{ij} | S_{ik}) \prod_{m \in O_k} Pr(o_{im} = 1) \prod_{n \in \bar O_k} Pr(o_{in} = 0) \\
&= \sum_{k:j \in O_k} \frac{\exp(\alpha w_{ij})}{\displaystyle\sum_{h \in O_k} \exp(\alpha w_{ih})} \prod_{m \in O_k, m > 0} \frac{\exp(\beta x_{i})}{1 + \exp(\beta x_i)} \\
&\times \prod_{n \in \bar O_k, n > 0} \frac{1}{1 + \exp(\beta x_i)}
\end{align}

Here, the term $Pr(o_{ij} = 1)$ represents the probability that country $j$ makes an offer to firm $i$, through which the official's preference enters our estimation. On the other side, $Pr(A_{ij}) | S_{ik}$ is the probability that firm $i$ will accept the offer from official $j$, given the offering set $O_k$. The firm's preference is reflected in our estimation through this term, $Pr(A_{ij})|S_{ik}$.\footnote{The appendix shows how these terms are derived.}

It is important to note that the offering set $O_k$ contains \textit{all} offers that firm $i$ receives, only one of which is the observed match between firm $i$ and country $j$. The intuition is that if we observe the full set of offers that all officials make to all firms, then by looking at the final match we can see how firms and officials reject inferior offers and thus deduce their preferences.

However, since the observed data only contains the final match, there is no information about the offers that firm $i$ receives but rejects, leading to the problem of incomplete data. To maximize the likelihood with incomplete data, we use the Expectation-Maximization (EM) algorithm, which runs as follows. First, we assume some arbitrary values for the parameters representing the actors' preferences. Second, given these preferences, we fill in the missing data, i.e. the offers that we do not observe. After this step, we now have the full set of offers and can estimate the actors' preferences again. These two steps constitute one iteration of the EM algorithm, which we repeat many times. 

Under certain regularity conditions, met by the TSL, the EM algorithm is proven to increase the likelihood after each iteration \citep[152]{Logan1998}. We stop the algorithm when the likelihood no longer increases and keep the final set of parameter estimates.

Depending on the starting values of the parameters, the EM algorithm may get stuck on a local maximum of the likelihood. Therefore, we will try multiple starting values and choose the one that leads to the highest likelihood.

\subsubsection{Hypothesis and Operationalization}
\label{sec:tsl_hypothesis}

As before, I hypothesize that an official with a longer time horizon would prefer spillover over private benefits. First, I exploit the variation in time horizon among democratic governments, operationalized as the level of institutionalization of the party in power. Highly institutionalized parties are long-lasting and encompass multiple generations of politicians. Since the young generation of politicians have their entire career attached to the party label, they have a long time horizon and consider the long-term effect of the government's policy. Therefore, they are likely to pressure or bribe their senior colleagues, i.e. those with short time horizon, into choosing FDI based on its spillover effect. In contrast, weakly institutionalized parties tend to be short-lived, unable to attract young politicians, and thus have a short time horizon.\footnote{Making a similar argument, \citet{Blake2013} shows that highly institutionalized parties sign less rigid bilateral investment treaties because they want to maintain flexibility in case conditions change during their long future reign.} Therefore, I hypothesize that:

\begin{hyp}
Democratic governments ruled by highly institutionalized parties prefer FDI with spillover more than those ruled by weakly institutionalized parties.
\end{hyp}

I operationalize party institutionalization with the age of party, as measured by the Database of Political Institutions \citep{Keefer2002}. In a presidential system, I consider the president's party to the party in power.

Second, I exploit the variation in time horizon among autocratic governments, operationalized as the probability of staying in power. Unlike democracies, autocracies only have infrequent, if not violent, removal of leaders, leaving little chance for the political losers to regain office. Therefore, once the probability of staying in power becomes small, the leader's time horizon is severely shortened. I hypothesize that:

\begin{hyp}
Autocratic regimes with higher probabilities of staying in power prefer FDI with high spillover.
\end{hyp}

I measure the probability of staying in power as the predicted probability of regime survival in a duration model, similar to  \citet{Wright2008a}.

Finally, to estimate the TSL model, we need data on the observed match between firms and countries / provinces, as well as their characteristics. I will use two datasets: 1) The Bureau of Economic Analysis' database of US Direct Investment Abroad (USDIA), and 2) Vietnam's Enterprize Census.