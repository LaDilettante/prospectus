\subsection{Project 3---time horizon of officials}

As discussed in \Cref{sec:theory_indifference_curve}, an official with a longer time horizon would choose a bundle with more spillover, whereas an official with a shorter time horizon would choose a bundle with more private benefits.

To get a handle on the time horizon of the official, we need to know the options provided to the official within the country's political economic system. Such is a difficult question to study with a cross-national design due to endogeneity issues, stemming from unobservable and unmeasurable differences across political systems. Therefore, at this step, I focus on the case of Vietnam, whose large number of provincial units (63) and their variation in FDI flow serve as an excellent testing ground. Again, here I focus on bribe and informal fees as the main form of private benefit for officials. Such constraint is not problematic for the case of Vietnam, where there is no campaign contribution and revolving door jobs are non-existent.

\subsubsection{The effect of time horizon on the choice of Vietnamese provincial officials} 

The relative weight assigned to spillover versus bribe by the Vietnamese provincial officials is determined by the principal-agent relationship between Vietnam's central and the provincial governments. On the one hand, the central government (i.e. the principal) cares more about the spillover effect of FDI and uses promotion to reward local officials that attract high-spillover FDI. On the other hand, local officials (i.e. the agent) have more opportunities to engage in corruption with foreign firms, and should they decide that the private benefit of corruption is greater than that of promotion, they will prioritize foreign firms that bring bribes over those that have high spillover effects.

The reason behind such difference in the preference of central and local governments is the fact that FDI projects are approved and managed at the provincial level. While the central law may be uniform in the book, its implementation varies widely across sub-national units in Vietnam \citep{Meyer2005}.\footnote{Vietnam's sub-national variation in implementation generalizes well to other cases, such as China \citep{Thun2006}} Therefore, provincial governments hold valuable services for sale to foreign firms. In contrast, the central government is not in charge of approving FDI projects (except those few with national importance) and thus less likely to benefit from corruption than provincial leaders.

In addition, the central government is much more concerned with overall economic growth, which is central to the longevity of the regime \citep{Malesky2008}. It wants to attract high spillover FDI and uses promotion to reward local officials that accomplish this goal. On the other hand, each provincial leader is incentivized to free-ride on the developmental effort of other provinces and of the central to keep the entire regime stable. Therefore, local officials value the spillover effect of FDI only insofar as the opportunities for promotion that it brings.

Fortunately for the central government, the principal-agent problem in this context is partially solved because monitoring is not too difficult. Indeed, the central government can observe the economic performance of the provinces and use personnel management to punish and reward provincial officials \citep{Sheng2007, Li2005}.\footnote{\citet{Shih2012} recently argue that economic performance does not matter to cadre promotion. However, they investigate all members of the Chinese Central Committee, including the central party apparatus, the army, and the central economic bureaucracy. These actors are not the important decision-makers in our theory.} Therefore, the principal-agent problem is only severe when provincial officials are not interested in further promotion to the central government, i.e. when the local official's time horizon is short. This suggests that there will be a variation in the level of FDI's spillover effect across provinces according to provincial officials' interest in promotion. In the research design, I use fuzzy regression discontinuity (RD) exploiting the mandated retirement age of Vietnamese officials, arguing that those that are in their last term have shorter time horizon and less interest in promotion.\footnote{The design is \textit{fuzzy} RD because the shortening of time horizon does not happen so abruptly as after a specific date. Instead, it happens in a time window after the official enters their last term before retirement.}

By looking at this variation in the career interest of provincial officials, my theory contributes a fresh angle to the current literature on the relationship between decentralization and corruption. So far, scholars have only postulated a one-way relationship: either decentralization increases bribery \citep{Fan2009} or reduces it \citep{Guerra2009}. In my model, how decentralization affects corruption is conditional on the local officials' interest in the promotions offered by the central government as carrots.


Three key assumptions in the theory above deserve further examination:
\begin{enumerate}
\item Why wouldn't Vietnam's central government worry that technological spillover would lead to a developed private sector, and consequently to social change that ultimately undermines its rule?

First, there is a large scholarship showing that authoritarian regimes are very adept at using institutions to manage regime outsiders in general and business in particular \citep{Gandhi2006, Gandhi2008, Wright2008, Le2015}. Second, if the legitimacy of the regime rests heavily on delivering economic growth, then the short-term risk of an economic downturn creating instability features much more prominently than the long-term concern with social changes. Third, it is possible to foster economic growth while restricting political freedom (e.g. Singapore). Indeed, growth can make a regime, both democratic and authoritarian, more stable, and creates room for political control \citep{Przeworski1997}.

\item Why don't provincial leaders seek rent from the domestic sector? 

First, Vietnam's private sector was very small when FDI was first allowed into Vietnam. The size and the profitability of the average domestic firm is still smaller than those of foreign firms today. Therefore, there are both fewer rents and more coordination problems if provincial officials want to seek rents from domestic firms. Second, ironically, if officials want to grow the private sector for future rent-seeking, they must promote an enabling business environment that are free from rent-seeking. In contrast, engaging in corruption with large and existing FDI firms is much more convenient. Essentially, corrupt provincial officials have shifted the cost of building a thriving domestic sectors to the home countries of FDI firms and now extract rents from the high productivity and high profitability of these firms. 
\end{enumerate}

In sum, I hypothesize that

\begin{hyp}
In provinces whose leaders are in their last term before the mandated retirement age, there are less spillover and more bribes from the foreign firms.
\end{hyp}

In addition to the fuzzy RD design using term limit, this project also attempts to measure the preference of the official directly with a survey conjoint analysis instead of relying on the observed level of bribe and spillover. Indeed, it is difficult to fully examine the official's utility function with only observational data because what he truly wants may not be fulfilled due to external and unobservable factors. Furthermore, what an official wants from a FDI firm is often hard to tease out completely. A big FDI firm is an attractive source of bribe, but it also brings job and technology. Indeed, perhaps this high multicollinearity is precisely why it is so easy for officials to extract bribe from FDI under the guise of promoting economic development.

To truly get at the utility calculation of provincial officials, I plan to conduct a survey experiment using conjoint analysis to ask provincial officials about their preference between two hypothetical FDI firms \citep{Hainmueller2014}. The characteristics of these firms will be randomly varied across several dimensions: size of labor force, capital, technology age, and most importantly, need for land, which proxies for corruption opportunities.

\subsubsection{Why choose land as a proxy for corruption?}

To discern provincial officials' preference for corruption opportunity versus developmental impact, one must vary the hypothetical FDI project along a characteristic that can only be attractive to officials because of its potential for corruption and not any other reasons. In this regard, the amount of land a project requires is the best proxy for corruption. Since land is an increasingly scarce and expensive resource in Vietnam, acquiring land from current tenants and farmers is a difficult, sometimes violent, process. Therefore, there is neither good developmental nor political reason for local officials to prefer a project that needs a large amount of land. 

In contrast, other characteristics of a FDI project can be preferred by officials for many different reasons. For example, a well capitalized project may signify a large pot of money to dip in, but it may also be attractive for the labor productivity enhancing effect of its capital. Similarly, a FDI firm with a large labor force may need to curry favor with officials to suppress their workers, but it may also be appealing for the jobs it creates.  Unlike those factors, land is unambiguously an indication of corruption opportunities. With a high level of \textit{monopoly} and \textit{discretion}, local officials are able to sell land access, something that investors are eager to buy.

1) Monopolistic control over land supply: At the start of Vietnam's liberalization (under Land Law 1993), any exchange of land between land users and investors must go through the local government. Investors had to negotiate with all levels of local governments (i.e. commune, district, and province level people's committees) to acquire land---a complex process that encouraged investors to use informal procedures and fees to expedite. Importantly, the price of land was solely determined by the local government, which was usually 10-30\% of the market price. Therefore, officials were able to extract bribes with both their gate-keeping and price-setting powers over land.

Subsequent land law reforms (2003 and 2013) attempted to bring the land acquisition process closer to a market approach and lessen the monopolistic control of the local government over land. For example, Land Law 2003 specified two methods for investors to acquire lands: voluntary and compulsory. Under compulsory land acquisition (akin to eminent domain), local governments retain the power to acquire land with compensation then allocate to approved investors. Under the newly-introduced voluntary land acquisition, investors negotiate with and buy from land users in a private market transaction. Despite the option of buying lands from private users, in practice most investment projects tellingly opted for compulsory land acquisition by the state. With the local government's coercive power and legal ability to set compensation value on their side, investors find compulsory land acquisition both faster and cheaper, and thus worth paying for.

Similarly, despite many calls for removing the state's control over land, Land Law 2013 disappointed with its insistence on ``people's ownership'' of land instead of adopting a fully private ownership system. Furthermore, the law preserves the state's right to acquire land for the vaguely defined ``socioeconomic development'' and ``national interest,'' which expansively includes the development of industrial zones.

2) Discretionary allocation of land to selected investors: Opportunities for corruption also arise from two discretionary powers of the local governments. First, land acquired by the government is allocated directly to approved investors instead of through public auction, an option allowed by law but rarely practiced by local governments. Second, in many cases, local officials even modify the existing land use plans according to the suggestions of investors, making available land that was previously not zoned for business development. Without any standard guideline for investor approval, this process relies heavily on personal contacts and is prone to bribery and kickback.

An important symptom of this corrupt practice is the lack of transparency in the land allocation process and decision. Key information, such as the criteria of project approval, the shortlist of investors, the profile of the selected projects and investors, and the (dictated) price of land, are kept among selected investors and a few state officials involved. Even a straightforward compliance with transparency regulation, i.e. the public posting of investment site maps, is not fulfilled. In a 2010 study, DEPOCEN researchers could only access the investment site maps in 2 of the 12 visited provinces \citep{Anderson2011}.\footnote{But Land law 2013 does remove the direct allocation of land to approved project, instead try to increase the number of land auctions. Does this have an effect?}

\subsubsection{Conjoint analysis design}
Please read the following description carefully. Then, please indicate which project you prefer to grant investment license (cap giay phep dau tu).

\begin{center}
  \begin{tabular}{ c | c | c }
    \hline
     & Project 1 (Du an 1) & Project 2 (Du an 1) \\ \hline
    Industry &  &  \\ \hline
    Labor force &  &  \\ \hline
    Capital &  &  \\ \hline
    Land &  &  \\ \hline
    Technology age &  &  \\ \hline
    \hline
  \end{tabular}
\end{center}

If you have to choose, which project do you prefer to grant investment license? Project 1 / Project 2

\begin{itemize}
\item Industry: textile, electronics, automobile, consumer product
\item Labor force: 5, 50, 100, 200, 500 employees
\item Capital:
\item Land:
\item Technology age: 
\end{itemize}