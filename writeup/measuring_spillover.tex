\subsection{Measuring the main dependent variable: spillover effect}
\label{sec:measure_spillover}

\subsubsection*{Measuring spillover indirectly}

Similar to the endogenization of technological change in growth theory, many researchers have investigated how technology spillover from FDI happens instead of assuming its inevitability \citep{Romer1994}. Several channels have been proposed, some of which suggest an indirect measure of technology spillover.

These channels are:
\begin{itemize}
	\item imitation:  private firms may reverse engineer a production or management technique \citep{Wang1992}, which is facilitated by backward linkage between local and foreign firms \citep{Javorcik2004}. This motivates my first measure of spillover effect: \% of private firms that participate in contracts with foreign firms.
	\item competition: similar to the effect of competition from arm's length trade on productivity, the presence of foreign firms in the domestic market put pressure on local firms to reduce inefficiency \citep{Glass2002}.
	\item export demonstration: foreign firms are more knowledgeable about exporting, which involves high fixed cost to set up a distribution and transport infrastructure, or learning about foreign taste and regulatory environment. Domestic firms can learn this ``export know-how'' from foreign firms \citep{Aitken1997}. This motivates my second measure of spillover effect: \% of domestic firms that export.
	\item skills acquisition: workers trained in foreign firms bring along their human capital when they move to domestic firms \citep{Djankov2000}. This presumes a healthy domestic sector that can offer competitive wages to workers.
\end{itemize}

Among these channels, the \textit{imitation} and \textit{export demonstration} channels form the theoretical basis of my two proxy measurements of spillover:
\begin{enumerate}
\item frequency of contracts between foreign and domestic firms,
\item percentage of domestic firms engaging in export.
\end{enumerate}

\subsubsection*{Measuring spillover directly}

As standard in the economic literature that studies whether there is a spillover effect for FDI, we can also measure spillover directly. This is done in two steps.

\begin{itemize}
\item First, measure the level of technology or productivity of a firm.

Level of technology: R\&D spending

Level of productivity \citep{VanBeveren2012}: Consider the familiar Cobb-Douglas production function:

\begin{align}
Y &= AL^{\alpha}K^{\beta}
\end{align}

where $Y$ is value added, $A$ is total-factor productivity (TFP), $L$ is labor, and $K$ is capital.\footnote{This is a structural equation and thus does not have an error term.} $y$, $L$, and $K$ are observable, while $A$ is not. Log transform both sides of the equation, we attain a linear form:

\begin{align} \label{eq:cobb_douglas_linear}
y &= a + \alpha l + \beta k
\end{align}

where the lowercase variables are the log-form of the uppercase variables (e.g. $y = \log(Y)$ and so on). \autoref{eq:cobb_douglas_linear} can then be estimated with OLS:

\begin{align} \label{eq:cobb_douglas_OLS}
y_i = \beta_0 + \beta_1 l_i + \beta_2 k_i + \epsilon_i
\end{align} 

where $\beta_0$ is the average TFP of all firms and $\epsilon$ is the firm-specific deviation from that mean. From the estimated coefficients of \autoref{eq:cobb_douglas_OLS}, we can estimate firm-level TFP as follows:

\begin{align}
a_i &= \hat\beta_0 + \hat\epsilon_i \\
A &= \exp^{\hat\beta_0 + \hat\epsilon_i}
\end{align}

\item Having estimated firm-level TFP (or technology), we then regress TFP (or technology) on the presence of FDI in a country / sector. FDI presence can be measured as:
\begin{itemize}
\item the proportion of output in a country (sector) that comes from foreign firms in that sector. This measure focuses on the horizontal, or intra-sectoral, linkage of FDI.
\item the proportion of input and output of a sector that comes from and to foreign firms (as measured in the input-output table). This measure focuses on the vertical, or inter-sectoral, linkage of FDI.
\end{itemize}
\end{itemize}

