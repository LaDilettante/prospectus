A closer look into the level of corruption across industries also suggests a negative relationship between spillover and bribe. According to the \citet{TransparencyInternational2011}'s Bribe Payer Index, which measures the propensity of firms from industrialized countries bribing abroad, the most corrupt sectors are public works contracts, utilities, real estate, oil and gas, and mining. These sectors are most vulnerable to corruption due to a lack of competition and a high level of government involvement. They also tend to be less conducive to spillover due to the winner-take-all structure of the industry \citep[138]{UNCTAD2001}. For example, in real estate, utilities, or mining, if scarce resources and contracts can only be won by large foreign firms, then these firms will capture the rent and perpetuate their market power while relegating local contractors to low-value added tasks.

In contrast, the least corrupt sectors are agriculture, light manufacturing, civilian aerospace, IT, and banking and finance. Among these, some are high tech industries (e.g. aerospace, IT) that governments may prioritize to facilitate spillover. Others have low barriers to entry and a divisible production process, both of which are conducive to domestic sourcing (e.g. agriculture, light manufacturing) \citep{TransparencyInternational2011}.

Beyond these broad strokes, the relationship between spillover and corruption emerges in more granularity within the same sector and across countries. For example, despite the stereotype as a high corruption, low spillover sector, the mining industries in Chile, Ghana, and Mozambique have substantial variation of spillover according to the host country's level of corruption. According to the \citet{TransparencyInternational2014}'s Corruption Perception Index, Chile, Ghana, and Mozambique rank 21, 61, and 119 out of 175 countries on control of corruption. Correspondingly, according to surveys of mining firms, Chilean foreign mines ``have the greatest proportion of domestic suppliers and carry out more valued-added activities in-country than [foreign mines] in Ghana and particularly more than in Mozambique'' \citep[127]{Farole2014}. The high level of FDI spillover in Chilean mining may be due to its developed economy and competent base of local suppliers. However, this does not explain the difference between Mozambique and Ghana, two countries with a similar level of GDP per capita.

Importantly, Ghana and Mozambique differ not only in the level of spillover, which can be influenced by economic forces outside the government's strategic decision. In addition, the two governments also diverge in their policies to promote supply chain linkages between foreign and domestic firms. In Ghana, the government worked with the private sector to develop regulations that put real teeth into the local content requirements in the Ghana Minerals and Mining Act (2006). According to these regulations, foreign mining firms are required to develop a five-year local procurement plan, including targets and strategies to develop domestic supplier capacity. These is also clear evidence that the government enforces these rules by striking at the profitability of the firms: when bids are within two percent of prices, the bid with the highest local content shall be selected. In stark contrast, Mozambique shows no commitment to local supplier development either in its Mining Law (2002) or Mining Regulations (Decree no.62/2006) \citep[137]{Farole2014}.

This example shows that if a government wants to induce spillover from foreign firms, it can. Even though policies that affect the profitability of the foreign firms, such as local content requirement, are technically not allowed under the national treatment principle of the WTO, there are many loopholes and little enforcement \citep{Hufbauer2013}. Indeed, Ghana itself has been a WTO member since 1995 and a GATT member since 1962. Therefore, the interesting question is not whether the government can extract spillover from FDI, but under what conditions would it want to do so.
