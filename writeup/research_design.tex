
\subsection{Measuring the main dependent variable: spillover effect}

\subsubsection*{Measuring spillover indirectly}

Similar to how growth economists start endogenizing technological change, FDI researchers investigate how technology spillover from FDI may happen instead of assuming its inevitability \citep{Romer1994}. Several channels for spillovers have been proposed, some of which suggest an indirect measure of technology spillover.

These channels are:
\begin{itemize}
	\item imitation:  private firms may reverse engineer a production or management technique \citep{Wang1992}, which is facilitated by backward linkage between local and foreign firms \citep{Javorcik2004}. This motivates my first measure of spillover effect: \% of private firms that participate in contracts with foreign firms.
	\item competition: similar to the effect of competition from arm's length trade on productivity, the presence of foreign firms in the domestic market put pressure on local firms to reduce inefficiency \citep{Glass2002}. (Doing Business has firm-level data on the number of private/state/foreign competitors in the last year)
	\item export demonstration: foreign firms are more knowledgeable about exporting, which involves high fixed cost to set up a distribution and transport infrastructure, or learning about foreign taste and regulatory environment. Domestic firms can learn this ``export know-how'' from foreign firms \citep{Aitken1997}. This motivates my third measure of spillover effect: \% of private firms that export.
	\item skills acquisition: workers trained in foreign firms bring along their human capital when they move to domestic firms \citep{Djankov2000}. This presumes a healthy domestic sector that can offer competitive wages to workers.
\end{itemize}

Among these channels, \textit{imitation} and \textit{export demonstration} forms the theoretical basis of my two proxy measurements of spillover:
\begin{enumerate}
\item frequent business contacts between foreign and domestic firms,
\item percentage of domestic firms engaging in export
\end{enumerate}

\subsubsection*{Measuring spillover directly}

As standard in the economic literature that studies whether there is a spillover effect for FDI, we can also measure spillover directly. This is done in two steps.

\begin{itemize}
\item First, measure the level of technology or productivity of a firm.

Level of technology: R\&D spending

Level of productivity: Consider the familiar Cobb-Douglas production function:

\begin{align}
Y &= AL^{\alpha}K^{\beta}
\end{align}

where $Y$ is value added, $A$ is total-factor productivity (TFP), $L$ is labor, and $K$ is capital. $y$, $L$, and $K$ are observable, while $A$ is not. Log transform both sides of the equation, we attain a linear form:

\begin{align} \label{eq:cobb_douglas_linear}
y &= a + \alpha l + \beta k
\end{align}

where the lowercase variables are the log-form of the uppercase variables (e.g. $y = \log(Y)$ and so on). \autoref{eq:cobb_douglas_linear} can then be estimated with OLS:

\begin{align} \label{eq:cobb_douglas_OLS}
y_i = \beta_0 + \beta_1 l_i + \beta_2 k_i + \epsilon_i
\end{align} 

where $\beta_0$ is the average total-factor productivity of all firms and $\epsilon$ is the firm-specific deviation from that mean. From the estimated coefficients of \autoref{eq:cobb_douglas_OLS}, we can estimate firm-level TFP as follows:

\begin{align}
a_i &= \hat\beta_0 + \hat\epsilon_i \\
A &= \exp^{\hat\beta_0 + \hat\epsilon_i}
\end{align}

\item Having estimated firm-level TFP (or technology), we then regress TFP (or technology) on the presence of FDI in a country / sector. FDI presence can be measured as:
\begin{itemize}
\item amount of FDI or number of foreign firms in a country (sector). This measure focuses on the horizontal, or intra-sector, linkage of FDI
\item number of foreign firms that the domestic firms are in contact with. This measure focuses on the vertical, or inter-sector, linkage of FDI
\end{itemize}
\end{itemize}



\subsection{Hierarchical model using cross-national, cross-sectoral data}

To measure corruption, presence of FDI, and treatment of firms across countries, I utilize the World Bank's Enterprise Survey (ES), which includes a wealth of firm-level data across 125 countries, spanning various topics from investment, labor, to business-government relation \citep{WorldBank2015}. The Enterprise Survey uses stratified random sampling (using three strata: firm size, business sector, and region) in order to ensure representativeness. The survey data comes from face-to-face interviews with upper management and is anonymized to ensure confidentiality at all times.\footnote{For more on the methodology of the Enterprise Survey, visit \url{http://www.enterprisesurveys.org/methodology}} This dataset has a wealth of firm-level data that helps us operationalize key concepts as detailed below.

Recall our hypothesis:

\begin{quote}
Hypothesis: The presence of large FDI firms in corrupt countries is associated with a low level of spillover effect in those countries.
\end{quote}

\begin{quote}
Hypothesis: The presence of large FDI firms in corrupt sectors is associated with a low level of spillover effect in those sectors.
\end{quote}

Operationalization of independent variables:
\begin{itemize}
\item FDI in countries: available via UNCTAD data on FDI flows and stocks to countries.

\item FDI in sectors: available via the Enterprises Survey dataset. The ``largeness'' of FDI firms can be measured by constructing a Herfindahl-Hirschman Index based on the size of sale, labor, or capital of firms. This allows us to calibrate the ``largeness'' of FDI firms according to the size of the host country's market.

\item Corruption: can be measured in two ways. 1) Firms' perception about corruption as an obstacle. This measure is frequently used but not accurate since firms' perception of corruption depends not only on the level of corruption but also the characteristics of firms. 2) Hard measure of prevalence and depth of bribes, e.g. ``Was an informal payment expected or request (when applying for a license)?'', ``How much do establishments like this one give in informal payments?'' 
\end{itemize}


\subsection{OECD ABC and List experiment to better measure corruption}

As mentioned earlier, despite the wealth of firm-level, cross-national data in the ES dataset, its measure of corruption is still plagued by a host of measurement issues. 

Asking directly about firms' experience with corruption is unlikely to get an accurate answer due to sensitivity bias \citep{Coutts2011}. Researchers, including the ES team, often address this problem by framing the question about the experience with corruption of ``firms like yours.'' However, with this technique, firms may not read between the lines and actually answer about the experience of others \citep{Ahart2004}.

I remedy these problems with a research design focusing on the case of Vietnam, taking advantage by a survey list experiment by \citet{Malesky2015}, which uses unmatched count technique to accurately measure the experience of firms with corruption while avoiding sensitivity bias.

Recall the hypothesis:

\begin{quote}
Hypothesis: The presence of large FDI firms in provinces whose leaders are not interested in promotion is associated with a low level of spillover effect.
\end{quote}

Operationalization of independent variables:
\begin{itemize}
\item FDI in province: provincial statistics of FDI flow (government website)
\item FDI in sectors: government website
\item Corruption: list experiment \citep{Malesky2015}
\item Interest in promotion: 
\begin{itemize}
	\item base chance of promotion: years until retirement (retirement age is 60 for male, 55 for female)
	\item appearance in centrally controlled newspapers as a proxy for the decision to pursue promotion
\end{itemize}
\end{itemize}


\subsection{Vietnam case study: addressing issues of endogeneity and measurement error of corruption}

In the cross national design, the corruption variable is problematic for two reasons. First, corruption, due to its complex institutional causes, is likely to be endogenous to a host of unobserved factors that also affect spillover. Second, a direct measure of corruption tends to suffer from sensitivity bias. In this section, I address both of these issues in turn, using 1) an exogenous shock to the level of corruption in Vietnam following the Phase 3 (Enforcement) of the OECD's anti-bribery convention in 2009, and 2) data from a list experiment by \citet{Malesky2015} to measure corruption.


\subsection{Conjoint analysis of Vietnamese officials' preference}

A crucial causal mechanism in my theory is the utility calculation of provincial officials, which weighs between the developmental impact and the potential for bribes of FDI. It is difficult to fully examine this key step with only observational data because what officials truly want may not be fulfilled due to external factors and thus cannot be observed. Furthermore, what an official wants from a FDI firm is often hard to tease out completely. A big FDI firm is an attractive source of rent, but it also brings job and technology. Indeed, perhaps this high correlation is precisely why it is so easy for officials to extract rent from FDI under the guise of promoting economic development.

To truly get at the utility calculation of provincial officials, I plan to conduct a survey experiment using conjoint analysis to ask provincial officials about their preference between two hypothetical FDI firms \citep{Hainmueller2014}. The characteristics of these firms will be randomly varied across several dimensions: size of labor force, capital, technology age, and most importantly, need for land, which proxies for corruption opportunities.

\subsubsection{Why choose land as a proxy for corruption?}

To discern provincial officials' preference for corruption opportunity versus developmental impact, one must vary the hypothetical FDI projects along a characteristic that can only be attractive to officials because of its potential for corruption and not because of any other reasons. In this regard, the amount of land a project requires is the best proxy for corruption. Since land is such a scarce resource with rapidly rising value in Vietnam, acquiring land from current tenants and farmers is a difficult, sometimes violent, process. Therefore, there is neither good developmental nor political reason for local officials to prefer a project that needs a large amount of land. 

In contrast, other characteristics of a FDI project can be preferred by officials for many different reasons. For example, a well capitalized project may signify a large pot of money to dip in, but it may also be attractive for the labor productivity enhancing effect of its capital. Similarly, a FDI firm with a large labor force may need to curry favor with officials to suppress their workers, but it may also be appealing for the jobs it creates.  Unlike those factors, land is unambiguously an indication of corruption opportunities. With a high level of \textit{monopoly} and \textit{discretion}, local officials are able to sell land access, something that investors are eager to buy.

1) Monopolistic control over land supply: At the start of Vietnam's liberalization (under Land Law 1993), any exchange of land between land users and investors must go through the local government. Investors had to negotiate with all levels of local governments (i.e. commune, district, and province level people's committees) to acquire land---a complex process that encouraged investors to use informal procedures and fees to expedite. Importantly, the price of land was solely determined by the local government, which was usually 10-30\% of the market price. Therefore, officials were able to extract bribes with both their gate-keeping and price-setting powers over land.

Subsequent land law reforms (2003 and 2013) attempted to bring the land acquisition process closer to a market approach and lessen the monopolistic control of the local government over land. For example, Land Law 2003 specified two methods for investors to acquire lands: voluntary and compulsory. Under compulsory land acquisition (akin to eminent domain), local governments retain the power to acquire land with compensation then allocate to approved investors. Under the newly-introduced voluntary land acquisition, investors negotiate with and buy from land users in a private market transaction. Despite the option of buying lands from private users, in practice most investment projects tellingly opted for compulsory land acquisition by the state. With the local government's coercive power and legal ability to set compensation value on their side, investors find compulsory land acquisition both faster and cheaper, and thus worth paying for.

Similarly, despite many calls for removing the state's control over land, Land Law 2013 disappointed with its insistence on ``people's ownership'' of land instead of adopting a fully private ownership system. Furthermore, the law preserves the state's right to acquire land for the vaguely defined ``socioeconomic development'' and ``national interest,'' which expansively includes the development of industrial zones.

2) Discretionary allocation of land to selected investors: Opportunities for corruption also arise from two discretionary powers of the local governments. First, land acquired by the government is allocated directly to approved investors instead of through public auction, an option allowed by law but rarely practiced by local governments. Second, in many cases, local officials even modify the existing land use plans according to the suggestions of investors, making available land that was previously not zoned for business development. Without any standard guideline for investor approval, this process relies heavily on personal contacts and is prone to bribery and kickback.

An important symptom of this corrupt practice is the lack of transparency in the land allocation process and decision. Key information, such as the criteria of project approval, the shortlist of investors, the profile of the selected projects and investors, and the (dictated) price of land, are kept among selected investors and a few state officials involved. Even a straightforward compliance with transparency regulation, i.e. the public posting of investment site maps, is not fulfilled. In a 2010 study, DEPOCEN researchers could only access the investment site maps in 2 of the 12 visited provinces \citep{Anderson2011}.\footnote{But Land law 2013 does remove the direct allocation of land to approved project, instead try to increase the number of land auctions. Does this have an effect?}

\subsubsection{Conjoint analysis design}
Please read the following description carefully. Then, please indicate which project you prefer to grant investment license (cap giay phep dau tu).

\begin{center}
  \begin{tabular}{ c | c | c }
    \hline
     & Project 1 (Du an 1) & Project 2 (Du an 1) \\ \hline
    Industry &  &  \\ \hline
    Labor force &  &  \\ \hline
    Capital &  &  \\ \hline
    Land &  &  \\ \hline
    Technology age &  &  \\ \hline
    \hline
  \end{tabular}
\end{center}

If you have to choose, which project do you prefer to grant investment license? Project 1 / Project 2

\begin{itemize}
\item Industry: textile, electronics, automobile, consumer product
\item Labor force: 5, 50, 100, 200, 500 employees
\item Capital:
\item Land:
\item Technology age: 
\end{itemize}
