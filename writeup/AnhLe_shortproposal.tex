\documentclass[12pt]{article}
% This first part of the file is called the PREAMBLE. It includes
% customizations and command definitions. The preamble is everything
% between \documentclass and \begin{document}.

\usepackage[margin=1in]{geometry}  % set the margins to 1in on all sides
\usepackage{graphicx}              % to include figures
\usepackage{amsmath}               % great math stuff
\usepackage{amsfonts}              % for blackboard bold, etc
\usepackage{amsthm}                % better theorem environments

\usepackage{rotating} % for sideway table
\usepackage{xcolor}
\usepackage{hyperref}
\hypersetup{
    colorlinks,
    linkcolor={red!50!black},
    citecolor={blue!50!black},
    urlcolor={blue!80!black}
}
\usepackage{cleveref} % Reference
\usepackage{enumitem} % nosep

\usepackage{array,tabularx}

\newenvironment{conditions*}
  {\par\vspace{\abovedisplayskip}\noindent
   \tabularx{\columnwidth}{>{$}l<{$} @{${}={}$} >{\raggedright\arraybackslash}X}}
  {\endtabularx\par\vspace{\belowdisplayskip}}
  
\usepackage{float}
\restylefloat{table}

% various theorems, numbered by section

\newtheorem{thm}{Theorem}[section]
\newtheorem{lem}[thm]{Lemma}
\newtheorem{prop}[thm]{Proposition}
\newtheorem{cor}[thm]{Corollary}
\newtheorem{conj}[thm]{Conjecture}
\newtheorem{hyp}{Hypothesis}

\DeclareMathOperator{\id}{id}

\newcommand{\bd}[1]{\mathbf{#1}}  % for bolding symbols
\newcommand{\RR}{\mathbb{R}}      % for Real numbers
\newcommand{\ZZ}{\mathbb{Z}}      % for Integers
\newcommand{\col}[1]{\left[\begin{matrix} #1 \end{matrix} \right]}
\newcommand{\comb}[2]{\binom{#1^2 + #2^2}{#1+#2}}

% bibliography
\usepackage{natbib}
\bibpunct{(}{)}{;}{a}{}{,} % no comma between author and year

\title{Prospectus: The political determinants of FDI technological spillover and corruption}
\author{Anh Le\\PhD candidate in Political Economy, Duke University\\\href{mailto:aql3@duke.edu}{aql3@duke.edu}}


\begin{document}
\maketitle

\section{The puzzle: why do countries offer investment incentives?}

Among the benefits that FDI brings to developing countries, technological spillover
is the most important factor to long-term economic growth. As well known from
growth theory, capital accumulation without
technological innovation 
will in the long run stop generating growth due to diminishing return \citep{Solow1956}.
This insight prompts scholars to argue that FDI is growth-enhancing
not so much because it brings capital, but because it leads to technological
spillover between foreign and domestic firms \citep{Nunnenkamp2004, Findlay1978}. In this view, FDI has a positive
externality, providing a boost in domestic firms' productivity that foreign
firms do not internalize while calculating their own benefits. This claim
about the public benefit of FDI justifies countries' use of investment incentives
to rectify the ``undersupply'' of FDI. 

However, it remains controversial whether investment incentives truly help. First,
incentives may not be effective in attracting FDI if multinational firms care a
lot more about fundamentals such as market, labor, or natural resources \citep{Blomstrom2002}. Second,
even if more FDI are attracted, there is little conclusive evidence of FDI
having a positive effect on growth \citep{Nair-Reichert2001, Carkovic2002} or
poverty reduction \citep{Guerra2009}. Indeed, FDI's
growth-enhancing effect is highly conditional on the absorptive capacity
of domestic firms. Investment incentives may be able to bring in FDI but do
little to improve the local absorptive capacity. More
insidious yet, handing out financial incentives deprives countries of revenue
they could have collected from multinationals, further curtailing their ability
to invest in improving local absorptive capacity.

Despite these caveats about the effectiveness of investment incentives, why do
many countries fixate on using them to attract FDI? To understand this puzzle, I propose that we need to take into account the
calculus of government officials, who may be interested in FDI as a source of private benefits rather than a driver of technological growth.

\section{The model: explaining FDI investment decision as an exchange between multinational firms and government officials}

In the model, the official has control over a certain endowment (e.g. market
access, cheap labor) that is attractive to multinational firms. Multinationals
who invest in the official's territory turn this endowment into profit via their
productive activities. Firms then share this profit with the official in
exchange for access to the endowment.  

Firms share the profit with the official in the form of a two-good bundle: 1)
technological spillover, and 2) private benefits (to the official). The official
wants technological spillover because it is a crucial ingredient in long-term
growth, which in turn, brings electoral or career benefits. The official also
wants the more direct private benefits, which can be both legal (e.g. campaign
contribution, informal network for revolving door employment) and illegal
(e.g. bribe, kickback).

How the firm and the official determine the mix of technological spillover and private
benefits depends on
several factors, which I will hypothesize and examine in each project below.

\subsection*{Hypothesis 1: A high cost of bribing increases the ratio of technological spillover of FDI}

To test this hypothesis, I use the OECD Anti-Bribery Convention (ABC) as a shock
in the cost of bribing. In December 2009, the OECD's Working Group on Bribery
(WGB) announced the Enforcement phase of the ABC, increasing both the
probability and the consequence of getting caught for firms from member
countries. Therefore, I argue that, following the Enforcement Phase of the Anti-Bribery
Convention, FDI firms from member countries will produce more technological
spillover and fewer bribes than firms from non-member countries.

We can test this argument in two ways.

{\color{red}First, we can examine investment incentives policy across countries before and after
ABC.} Before ABC, the cost of bribing is low---thus, we would expect officials to
offer more investment incentives to attract FDI and share in the rent. By
contrast, after ABC, firms will be more hesitant to offer bribes, making
officials less enthusiastic about offering incentives. In addition, after ABC,
we expect that the type of FDI attracted has more potential for technological
spillover. Important for our research design, we
expect to see this change in countries whose main FDI source is firms
from ABC member countries, but not in countries whose main FDI source is firms
from ABC non-member countries. This suggests a difference-in-difference research design.

Second, we can focus on a case study, for which Vietnam is an ideal setting for
three reasons. First, Vietnam attracts
FDI from both member and non-member countries of the ABC. Second, given that
data on bribe and corruption is often unreliable, Vietnam is unique in the availability of a FDI
bribery data, provided by the list experiments run in its national annual business survey
\citep{Malesky2015}. Using the unmatched count technique, the list experiment
allows respondents to be honest in a survey
about bribes without incriminating themselves.

\subsection*{Hypothesis 2: Officials with a short time horizon prefer FDI with
  more private benefits}

Because technological spillover takes a long time to bear fruits, government
officials with a short time horizon will prefer FDI projects that bring
immediate private benefits to them. I test this hypothesis in two projects.

{\color{red}First, cross-nationally, I use the two sided matching model to examine the
preferences of firms and countries for one another. Originally developed for the
labor market and the marriage market, the matching model allows us to say
what kind of firms like what kind of countries, and vice versa. Therefore, we
can test whether officials with a short time horizon (e.g. when their
re-election chance is in jeopardy) prefer FDI with a high corruption tendency
over FDI with a high spillover potential. Conversely, we can test whether FDI firms
of a certain industry prefer countries with high labor quality, stable
governments, or proximity to markets.} This model makes use of the
recent availability of global firm level data, which allows us to track FDI
investment more accurately than country level FDI flow of the past. Indeed,
since aggregate FDI flow is calculated as a by-product of the Balance of Payment
accounting, it perfectly captures the flow of capital on the book, but not
necessarily the change in real factors that we care about (e.g. jobs created
or fixed assets investment).

Second, I take advantage of Vietnam's mandated retirement age in a regression
discontinuity design. If a Vietnamese official is past the age of 60, he or she
is serving their last term before retirement, and thus no longer has the chance
to be promoted higher. Such officials will have a short time horizon, placing
less value on the long-run benefit of technological spillover. Therefore, we can
compare the mix of FDI attracted by officials just above and below the age-of-60
threshold. According to our hypothesis, officials younger than 60 years will
focus more on spillover and less on private benefits than officials older than
60 years.

\clearpage
\bibliographystyle{chicago}
\bibliography{library}

\end{document}