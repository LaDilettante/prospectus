First, \Cref{sec:rd_measure_spillover} demonstrates how to measure FDI spillover, the key outcome variable of the project. Then, the next three subsections test the effect of three key parameters in the theoretical model. We want to see how each of these parameters affect the mix of spillover and private benefits that the official chooses. \Cref{sec:rd_potential_for_spillover} examines the first parameter---the ``price'' of spillover---arguing that the official extracts more spillover and fewer bribes from firms that have a high potential for spillover. \Cref{sec:rd_private_benefit} examines the second parameter---the ``price'' of private benefit---arguing that the OECD Anti bribery convention raises the cost of bribing for firms from member countries. Therefore, the official will extract fewer bribes and more spillover from these firms. \Cref{sec:rd_time_horizon} examines the third parameter---the official's time horizon. I argue that if the official has a short time horizon (e.g. a Vietnamese provincial official facing term limit), he will extract more bribes and less spillover from firms.

Finally, \Cref{sec:rd_tsl} uses the two-sided logit (TSL) model to extend the main theoretical model in \Cref{sec:theory} by including both the official and the firm as strategic actors. I also generalize the time horizon argument to a cross-national setting and test whether a government's longer time horizon leads to a higher preference for spillover.