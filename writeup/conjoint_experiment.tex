\begin{quote}
Two FDI projects want to enter your province. Please carefully read the following description of the projects. Then, please indicate which project you prefer.

\begin{center}
  \begin{tabular}{ c | c | c }
    \hline
     & Project 1 (Du an 1) & Project 2 (Du an 1) \\ \hline
    Industry &  &  \\ \hline
    Labor force &  &  \\ \hline
    Capital &  &  \\ \hline
    Land &  &  \\ \hline
    Technology age &  &  \\ \hline
    \hline
  \end{tabular}
\end{center}

If you have to choose, which project do you prefer to grant investment license? Project 1 / Project 2
\end{quote}

The five dimensions will be given random values as follows.
\begin{itemize}
\item Industry: textile, electronics, automobile, consumer product
\item Labor force: 5, 50, 100, 200, 500 employees
\item Capital:
\item Land:
\item Technology age: 
\end{itemize}

If desired, it is possible to:
\begin{itemize}
\item adjust the design so that implausible hypotheticals will not appear (i.e. there should not be a high-tech company with very small capital).
\item randomize the ordering of the characteristics between respondents to test for the ordering effect (i.e. knowing a firm's industry first changes how the respondent thinks about the other characteristics)
\end{itemize}

I am mainly interested in the ``average marginal component effect'' (AMCE) of \textit{land}, which is the marginal effect of \textit{land} on the likelihood of a project being preferred, averaged over the distribution of all the other components. This allows us to back-out what provincial officials truly want from a FDI project.
